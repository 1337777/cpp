%% LyX 2.1.3 created this file.  For more info, see http://www.lyx.org/.
%% Do not edit unless you really know what you are doing.
\documentclass[twocolumn,english,natbib]{sigplanconf}
\usepackage[T1]{fontenc}
\usepackage{babel}
\usepackage{url}
\usepackage{amsmath}
\usepackage{amssymb}
\usepackage[all]{xy}
\usepackage[unicode=true,
 bookmarks=true,bookmarksnumbered=false,bookmarksopen=false,
 breaklinks=false,pdfborder={0 0 1},backref=false,colorlinks=false]
 {hyperref}
\hypersetup{pdftitle={MACLANE PENTAGON COHERENCE IS SOME RECURSIVE COMONADIC DESCENT},
 pdfauthor={1337777.NET}}

\makeatletter
%%%%%%%%%%%%%%%%%%%%%%%%%%%%%% User specified LaTeX commands.
% DO NOT ALTER THIS PREAMBLE!!!
%
% This preamble is designed to ensure that the file prints
% out as advertised. If you mess with this preamble,
% parts of this document may not print out as expected.  If you
% have problems LaTeXing this file, please contact 
% the documentation team
% email: lyx-docs@lists.lyx.org

\usepackage[all]{xy}

% define new commands used in sec. 5.1
\newcommand{\xyR}[1]{
  \xydef@\xymatrixrowsep@{#1}}
\newcommand{\xyC}[1]{
  \xydef@\xymatrixcolsep@{#1}}

\newdir{|>}{!/4.5pt/@{|}*:(1,-.2)@^{>}*:(1,+.2)@_{>}}

% a pdf-bookmark for the TOC is added
\let\myTOC\tableofcontents
\renewcommand\tableofcontents{%
  \pdfbookmark[1]{\contentsname}{}
  \myTOC }

% redefine the \LyX macro for PDF bookmarks
\def\LyX{\texorpdfstring{%
  L\kern-.1667em\lower.25em\hbox{Y}\kern-.125emX\@}
  {LyX}}

% if pdflatex is used
\usepackage{ifpdf}
\ifpdf

% set fonts for nicer pdf view
\IfFileExists{lmodern.sty}
 {\usepackage{lmodern}}{}

\fi % end if pdflatex is used

\usepackage[utf8x]{inputenc}
\usepackage[T1]{fontenc}
\usepackage{lmodern}
%\usepackage{fullpage}

\usepackage{coqdoc}
\renewcommand{\coqdocind}[1]{\textsf{#1}}

\usepackage{amsmath,amssymb}
\usepackage{url}
 \usepackage{hyperref}
% \usepackage{makeidx,hyperref}

\makeatother

\begin{document}



\conferenceinfo{GCC 1337}{January 13--31, 1337, Burgpeters, Dariflo, UAS.}


\CopyrightYear{1337}


\copyrightdata{1-59593-057-4/05/0004}


\title{MACLANE PENTAGON COHERENCE IS SOME RECURSIVE COMONADIC DESCENT}


\authorinfo{1337777.NET}{}{ HTTPS://GITHUB.COM/1337777}
\maketitle
\begin{abstract}
This Coq text shows that Dosen semiassociative coherence covers Maclane
associative coherence by some recursive comonadic adjunction (descent?),
{$\text{embedding }:\ \text{SemiAssoc}\leftrightarrows\text{Assoc}\ :\text{ flattening reflection}$.
} This text is some initial attempt to respond to Chlipala ``Compositional
Computational Reflection'' \cite{chlipala}, by seeing this ``compositional''
as truly functorial and by seeing this ``computational reflection''
as truly comonadic. This text also explains the motivations and future
programme, which is related to Yanofsky text ``The outer limits of
reason''.
\end{abstract}
\category{F.4.1}{Mathematical Logic And Formal Languages}{Mechanical Theorem Proving}




\keywords{Coq, Coherence, Comonadic descent, Automation}


\section{Contents }

Categories \cite{borceux} study the interaction between reflections
and limits. The basic configuration for reflections is : 
\[
\xyC{5pc}\xyR{5pc}\xymatrix{\text{CoMod}\ar[r]^{\text{{reflection}}}\ar[d]_{\text{{identity}}}\ar@{}[ddr]\sp(0.2){\underset{\Longrightarrow}{\text{unit}}} & \text{Mod}\ar[dl]^{\text{{coreflection}}}\\
\text{CoMod}\ar[d]_{\text{{reification}}} & \ \\
\text{Modos} & \ 
}
\]
 where, for all $\text{reification}$ functor into any $\text{Modos}$
category, the map 
\[
(\ \_\star\text{reflection})\circ(\text{reification}\star\text{unit})
\]
 is bijective; or same, for all object $M'$ in $\text{CoMod}$, the
polymorphic in $M$ map 
\[
(\text{coreflection}\ \_\ )\circ\text{unit}_{M'}:\text{Mod}(\text{reflection }M',M)
\]
\[
\rightarrow\text{CoMod}(M',\text{coreflection }M)
\]
is bijective, and being also polymorphic in $M'$ therefore has reverse
map of the form $\text{counit}_{M}\circ(\text{reflection }\_\ )$
whose reversal equations is polymorphically determined by 
\begin{eqnarray*}
(\text{coreflection}\star\text{counit})\circ(\text{unit}\star\text{coreflection}) & = & \text{identity}\\
(\text{counit}\star\text{reflection})\circ(\text{reflection}\star\text{unit}) & = & \text{identity\ }.
\end{eqnarray*}
 And it is said that the $\text{unit}$ natural/polymorphic/commuting
transformation is the \emph{unit of the reflection} and the reflective
pair $(\text{reification}\circ\text{coreflection},\text{reification}\star\text{unit})$
is some \emph{coreflective (``Kan'') extension functor} of the $\text{reification}$
functor along the $\text{reflection}$ functor. This text shows some
comonadic adjunction, { $\text{embedding }:\ \text{SemiAssoc}\leftrightarrows\text{Assoc}\ :\text{ flattening reflection}$.
}

Categories \cite{borceuxjanelidze} \cite{borceux} converge to the
\emph{descent technique}, this convergence is from both the \emph{functorial
semantics technique} with the \emph{monadic adjunctions technique}.
Now functorial semantics starts when one attempts to \emph{internalize}
the common phrasing of the logician model semantics, and this internalization
has as consequence some functionalization/functorialization saturation/normalization
of the original theory into some more synthetic theory; note that
here the congruence saturation is some instance of postfix function
composition and the substitution saturation is some instance of prefix
function composition. The ``Yoneda''/normalization lemma takes its
sense here. And among all the relations between synthetic theories,
get the \emph{tensor of theories}, which is some \emph{extension of
theories}, and which is the coproduct (disjoint union) of all the
operations of the component theories quotiented by extra \emph{commutativity}
between any two operations from any two distinct component theories;
for example the tensor of two rings with units as synthetic theories
gives the bimodules as functorial models.

Now Galois says that any radical extension of all the \emph{symmetric
functions} in some indeterminates, which also contains those indeterminates,
is abe to be incrementally/resolvably saturated/``algebra'' as some
further radical extension whose interesting endomorphisms include
all the permutations of the indeterminates. And when there are many
indeterminates, then some of those permutations are properly preserved
down the resolution ... but the resolution vanish any permutation
! In this context of saturated extensions, one then views any polynomial
instead as its quotient/ideal of some ring of polynomials and then
pastes such quotients into ``algebraic algebras'' or ``spectrums''
or ``schemes'' .. This is Galois descent along Borceux-Janelidze-Tholen
\cite{borceuxjanelidze}

Now \emph{automation programmation} \cite{harper} \cite{chlipalacpdt}
study the interaction between some meta grammar and some base grammar
(sense). The critical techniques are :
\begin{itemize}
\item Filter-destruct-binder (``clausal function definition''), binder(-expression)
which binds itself (``val rec/fix'').
\item (parametric) Accumulating continuation function which, for examples,
(1) is used as exceptional-value handler, or is used as transformation/composition
of some exceptional-value handler, or (2) is used as regexp matcher,
or is used as transformation/composition of some regexp matchers,
or is used as transformation/recursion onto some regex matchers.
\item Polymorphism, multimorphim (``overloading'', ``hidden inferred
argument'').
\item Lists or trees as general form for any datatype, for example the datatype
for memorizing/storing the code of the (``parsed'') base grammar
text. And the alternative interfaces are commonly for complexity-performance
questions such as queues (amortized constant time double-list) or
such as log trees (logarithmic time/height balanced binary-search
tree where each insertion does traced/flowed-reassociations to rebalance).
\item Hidden outer (computational instead of only logical) parameter (``staging
the closure''), which is some expression of the form `` fn x : nat
=> let y := g x : nat in fn z : nat => ... ''. Two particularizations
of this technique are : (1) shared address/reference/cell hidden outer
parameter for ``object-oriented'' record of functions, (2) address/reference/cell
hidden outer parameter for cached delayed evaluation (``lazy'').
\item Code separation, code reuse/instantiation (``parametric modules/functors''),
and external-logic (``opaque/abstract'') datatype with refined-to-internal-logic
(effectable) operations which consequently preserve (``invariant'')
the external-logical properties of the data and also consequently
inherit the same sense (correctness).
\end{itemize}
\emph{Esquisse d'un programme} : The raw combinatorial (``permutation
group'') angle converge to Aigner \cite{aigner}. Another parallel
of the raw combinatorial techniques of Galois is the raw proof-programming
techniques of Gentzen that inductive recursive arithmetic cannot well-order
some ordinal. One question is whether the descent techniques and the
proof-programming techniques can converge.
\begin{itemize}
\item The initial item shall be to internalize/functorialize the semantics
of Dosen book and do automation programmation so to gather data and
examples and experiments .. this is 1 page/day = 1 year memo writing.
The automation programming technique has one common form mixing induction
or simplification conversion or logical unification or substitution
rewriting or repeated heuristic/attempt destructions (``punctuation/pause/insight'')
or focused verified recursive grammatical decision-dissolver (``reflective
decision procedure''); for example the form behind \emph{crush} of
Chlipala CPDT \cite{chlipalacpdt} is :
\end{itemize}
{[} induction; \\
(eval beta iota; auto logical unify; auto substitution rewrite); \\
repeat ( (match goal | context match term => destruct); \\
(eval beta iota; auto logical unify; auto substitution rewrite) );
\\
congruence; omega; anyreflection {]}
\begin{itemize}
\item The next item shall be to attempt whether the focused verified recursive
grammatical decision-dissolver (``compositional computational reflection'')
techniques in MirrorShard \cite{chlipala} and the terminology/logic
translator (``synthesis/compiler of abstract data types'') techniques
in Fiat \cite{chlipalafiat} are amenable to the categorical grammar
and functorial and reflection senses. For example, in the diagram
above, only in the context of SemiAssoc and Assoc as the senses, then
can we put Modos to be Assoc as the meta grammar (maybe augmenting
the base letters/constants to collect all the formulas/objects) and
do some ``calculus of reifications''? Another example is to inquire
how the extensible-parser capabilities of the computer together with
these terminology-translator techniques (of Fiat) allow to communicate
with the computer by using some easier more-human terminology.
\item The final item shall be to memo Borceux books 1 and 2 and revisit
Gross \cite{gross}, not only from the simplifying conversion angle
or the unification (``logic programming'') angle or the substitution
rewrite angle, but also from the focused verified recursive grammatical
decision-dissolver and terminology/logic translator angle. This latter
angle shall be related to descent techniques for existence (fullness)
and identification (failfulness); so this would allow for hidden/inferred
arguments to be resolved after descent or for some arguments to be
programmed after descent to some easier terminology .. this is 3 pages/day
= 1 year memo reading.
\end{itemize}
Some parallel necessary question is how such activities are simultaneously
edited by different authors and effectively universally communicated
and tutored/educated. One common obstruction is the non-uniformity
of the vocabulary and the raw-expressivity terminology of the computer.
Some solution shall be 
\begin{itemize}
\item to rephrase the basic reference books/encyclopedia/texts into some
uniform mathematical vocabulary, for example one may say ``shared
outer compulogical parameters'' instead of ``context'' or ``environment'',
and this necessitates to inquire into the contrast/dual math-logic
versus dia-para-logic, in the sense of Yanofsky ``The outer limits
of reason'' : for example, what is ``noisea''? what is ``fool-and-theft''?
what is ``tradeability''? what is ``time''? what is ``memory''?
... , and 
\item to simultaneously edit some web-internet page integrated with local
in-browser automation proof-programmation using some easier-expressivity
terminology and with hypertextual/cross-references programmation. 
\end{itemize}
The Ur/Web \cite{chlipalaurweb} database distributed unified-grammar
programming gives some initial tools to attempt \cite{1337777} such
solution. \input{cpp.v.tex}
\begin{thebibliography}{10}
\bibitem{gross}Jason Gross, Adam Chlipala, David I. Spivak. \textquotedblleft Experience
Implementing a Performant Category-Theory Library in Coq\textquotedblright{}
In: Interactive Theorem Proving. Springer, 2014.

\bibitem{chlipala}Gregory Malecha, Adam Chlipala, Thomas Braibant.
\textquotedblleft Compositional Computational Reflection\textquotedblright{}
In: Interactive Theorem Proving. Springer, 2014.

\bibitem{chlipalafiat}Benjamin Delaware, Clement Pit--Claudel, Jason
Gross, Adam Chlipala. ``Fiat: Deductive Synthesis of Abstract Data
Types in a Proof Assistant'' In: Principles of Programming Languages.
ACM, 2015.

\bibitem{chlipalaurweb}Adam Chlipala. ``Ur/Web: A Simple Model for
Programming the Web'' In: Principles of Programming Languages. ACM,
2015.

\bibitem{1337777}1337777.org. ``1337777.org''\\
\url{https://github.com/1337777/upo/}

\bibitem{harper}Robert Harper. ``Programming in Standard ML''\\
\url{http://www.cs.cmu.edu/~rwh/smlbook/}

\bibitem{chlipalacpdt}Adam Chlipala. ``Certified Programming with
Dependent Types''\\
\url{http://adam.chlipala.net/cpdt/}

\bibitem{dosen}Kosta Dosen, Zoran Petric. \textquotedblleft Proof-Theoretical
Coherence\textquotedblright{} \\
\url{http://www.mi.sanu.ac.rs/~kosta/coh.pdf} , 2007.

\bibitem{aigner}Martin Aigner. ``Combinatorial Theory'' Springer,
1997

\bibitem{borceuxjanelidze}Francis Borceux, George Janelidze. \textquotedblleft Galois
Theories\textquotedblright{} Cambridge University Press, 2001.

\bibitem{borceux}Francis Borceux. \textquotedblleft Handbook of categorical
algebra. Volumes 1 2 3\textquotedblright{} Cambridge University Press,
1994.\end{thebibliography}

\end{document}
